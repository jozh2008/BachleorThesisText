\documentclass[class=scrbook, crop=false]{standalone}
\usepackage[subpreambles=true]{standalone}
\ifstandalone
    \input{../settings+/settings}
\fi

% ----------------------------------------------------------------------------
%                               Theoretical Background
% ----------------------------------------------------------------------------
\begin{document}

\ifstandalone
    %\selectlanguage{ngerman}  % Toggle ON/OFF

    % Language-specific settings that change automatically
    \input{settings+/language}
\fi

\chapter{Theoretical Background}
\label{Chapter::Theoretical_Background} % Outline text
This chapter provides background knowledge to understand the concepts of Open Geospatial Consortium standards and gives you an introduction to the ZOO-Project.


\section{ZOO-Project}
\label{Section:: ZOO-Project}

\subsection{What is ZOO-Project?}
ZOO-Project is a WPS (Web Processing Service) implementation written in C, Python and JavaScript. It is an open source platform which implements the WPS 1.0.0 and WPS 2.0.0 standards edited by the Open Geospatial Consortium (OGC).
It provides the polyglot ZOO-Kernel, a server implementation of the Web Processing Service (WPS) (1.0.0 and 2.0.0) and the OGC API - Processes standards published by the OGC, https://zoo-project.github.io/docs/intro.html



\subsubsection{WPS (Web Processing Service)}
The OpenGIS® Web Processing Service (WPS) Interface Standard provides rules for standardizing how inputs and outputs (requests and responses) for geospatial processing services, such as polygon overlay. The standard also defines how a client can request the execution of a process, and how the output from the process is handled. It defines an interface that facilitates the publishing of geospatial processes and clients’ discovery of and binding to those processes. The data required by the WPS can be delivered across a network or they can be available at the server.

WPS is a standardized protocol for publishing geospatial processes and algorithms over the web, allowing users to access and execute these processes remotely.

https://www.ogc.org/standard/wps/

ZOO-Project provides a developer-friendly framework for creating and chaining WPS compliant Web Services. Its main goal is to provide generic and standard-compliant methods for using existing open source librairies and algorithms as WPS. It also offers efficient tools for creating new innovative web services and applications.

ZOO-Project is able to process geospatial or non geospatial data online. Its core processing engine (aka ZOO-Kernel) lets you execute a number of existing ZOO-Services based on reliable software and libraries. It also gives you the ability to create your own WPS Services from new or existing source code, which can be written in seven different programming languages. That lets you compose or turn code as WPS Services simply, with straightforward configuration and standard coding methods.

ZOO-Project is very flexible with data input and output so you can process almost any kind of data stored locally or accessed from remote servers and databases. ZOO-Project excels in data processing and integrates new or existing spatial data infrastructures, as it is able to communicate with map servers and can integrate webmapping clients.
https://zoo-project.github.io/docs/intro.html



The ZOO-Project is an open-source software framework that provides a platform for building and deploying geospatial web services. While the Open Geospatial Consortium (OGC) develops standards for geospatial technologies, the ZOO-Project implements these standards and provides a framework for creating web services that adhere to OGC specifications.

Here are some key differences:

    Scope: OGC is an international consortium focused on developing and promoting standards for geospatial technologies, covering various aspects such as data formats, web services, and metadata. The ZOO-Project, on the other hand, is a specific software framework for building and deploying geospatial web services.

    Role: OGC sets the standards that define how geospatial data and services should be structured and shared. The ZOO-Project implements these standards and provides a platform for creating web services that conform to OGC specifications.

    Functionality: OGC standards define protocols and data formats for interoperability between different geospatial systems and applications. The ZOO-Project, meanwhile, provides tools and libraries for developing custom geospatial web services, such as spatial data processing, geoprocessing, and spatial analysis.



    

% Background topics that are necessary to understand your thesis
\section{OGC Standards}
\label{Section::OGC Standards}

The onset of the Information Age and Digital Revolution has created a knowledge based society where the internet acts as a global platform for the sharing of information. In a geospatial context, this resulted in an advancement of techniques in how we acquire, study and share geographic information and with the development of Geographic Information Systems (GIS), locational services, and online mapping, spatial data has never been more abundant. The transformation to this digital era has not been without its drawbacks, and a forty year lack of common polices to data sharing has resulted in compatibility issues and great diversity in how software and data are delivered.

Essential to the sharing of spatial information is interoperability, where different programmes can exchange and open data from various sources seamlessly. Applying universal standards across a sector provides interoperable solutions. The Open Geospatial Consortium (OGC) facilitates interoperability by providing open standard specifications which organisations can use to develop geospatial software. This means that two separate pieces of software or platforms, if developed using open standard specifications, can exchange data without compatibility issues. By defining these specifications and standards the OGC plays a crucial role in how geospatial information is shared on a global scale.

Standard specifications are the invisible glue that holds information systems together, without which, data sharing generally would be an arduous task. On some level they keep the world spinning and this course will instil some appreciation for them from a geospatial perspective.

This course introduces users to the OGC and all the common standards in the context of geoportals and mapping solutions. These standards are defined and explored using a number of platforms and interoperability is demonstrated in a practical sense. Finally, users will implement these standards to develop their own platforms for sharing geospatial information.

https://learningzone.rspsoc.org.uk/index.php/Learning-Materials/Introduction-to-OGC-Standards/Introduction-to-OGC-Standards

\section{What is the OGC?}
OGC is a free and open geospatial Standards define interoperable approaches to Data Encoding, Data Access, Data Processing, Data Visualization, and Metadata and Catalogue Services.  OGC Standards lie at the heart of FAIR (Findable, Accessible, Interoperable, and Reusable) geospatial information. For three decades now, OGC Standards have been used by thousands of organizations across the globe to ensure interoperability and maximize the value of their geospatial data. Developed through consensus, and backed by government and organizations across the globe, OGC Standards provide the stable platform upon which geospatial innovation is built. 

OGC Standards serve as the bedrock for FAIR geospatial information—Findable, Accessible, Interoperable, and Reusable.
These standards have been adopted by thousands of organizations worldwide to ensure interoperability and maximize the value of their geospatial data.


OGC standards are specifications for open interfaces, protocols, schemas and so forth that enable different vendors' systems to exchange geospatial data and instructions, and that enable full integration of these capabilities into a variety of information systemhttps://www.ogc.org/standards/ 

The collection of geoportals and various other compliemntary services, create a Spatial Data Infrastructure (SDI).
https://learningzone.rspsoc.org.uk/index.php/Learning-Materials/Introduction-to-OGC-Standards/1.1-What-is-the-OGC

Benefits of the OGC

    Interoperability of geospatial data and reduced fragmentation in data delivery.
    Consensus based approach. Participation of organisations from the public sector, private sector, academia and research when developing standards assures the interests and needs of the geospatial community are considered.
    OGC helps to bring together geospatial data and services from multiple sectors (Figure 1.6).
    https://learningzone.rspsoc.org.uk/index.php/Learning-Materials/Introduction-to-OGC-Standards/1.5-Why-is-the-OGC-important



What is a standard?

In the OGC context, a standard is an agreed specification of rules and guidelines about how to implement software interfaces and data encodings. Geospatial software vendors, developers and users collaborate in the OGC’s consensus process to develop and agree on standards that enable information systems to exchange geospatial information and instructions for geoprocessing. OGC standards are open standards.


Open Standards

Organizations like the OGC, the IETF, the World Wide Web Consortium (W3C) and others are open organizations in the sense that any individual or organization can participate, the topics of debate are largely public, decisions are democratic (usually by consensus), and specifications are free and readily available. An “open” process is necessary to arrive at an “open” standard. The openness that OGC promotes is part of this general progress.

Often the terms “open standards” and “open source” are confused or incorrectly taken to mean the same thing. The OGC standards are specifications developed in an open process. Open source is software made freely available under a license that allows the program to run for any purpose, to study how the program works, to adapt it, and to redistribute copies, including modifications. As a matter of policy, the OGC Board of Directors and staff don’t favor either proprietary software or open source software. From the OGC perspective, any developer who implements OGC standards in software or online services is doing the right thing. OGC cares about interoperability – the ability to share geospatial information.
Open Standards - the definition

The OGC defines Open Standards as standards that are:

    Freely and publicly available – They are available free of charge and unencumbered by patents and other intellectual property.

    Non discriminatory – They are available to anyone, any organization, any time, anywhere with no restrictions.

    No license fees - There are no charges at any time for their use.

    Vendor neutral - They are vendor neutral in terms of their content and implementation concept and do not favor any vendor over another.

    Data neutral – The standards are independent of any data storage model or format.

    Based on Consensus - They are defined, documented, and approved by a formal, member driven consensus process. The consensus group remains in charge of changes and no single entity controls the standard.

An “Open Standard” is not the same as “Open Source”. “Open Source” refers to “Free and Open Source Software”, which has been made available under a free software license with the rights to run the program for any purpose, to study how the program works, to adapt it, and to redistribute copies, including modifications. These freedoms enable Open Source software development, a public, collaborative model that promotes early publishing and frequent releases. Most open source software products implement open standards, such as OGC standards. Some are also reference implementations of OGC. A reference implementation is an example of correct implementation of a standard, for use by developers that is free and publicly available for testing via a web service or download.


https://opengeospatial.github.io/e-learning/ogc-standards/text/services-ogc.html

    \subsection{Who is part of the OGC?}

    \subsection{Example of Glossary}
    \gls{Biofouling}
    \gls{Interoperability}

    \subsection{Example of Symbols}
    Examples of the use of symbols in a document. Symbols and also acronyms are introduced in a table at the begin of the document. To jump to this table, simply click on the symbol or acronym, it provides a hyperlink to the table.\\
    \gls{symb:height}
    \gls{symb:energy}
    \gls{symb:A}
    \gls{symb:B}
    \gls{symb:C}
    \gls{symb:D}
    \gls{symb:E}
    \gls{symb:F}
    \gls{symb:G}
    \gls{symb:H}
    \gls{symb:I}
    \gls{symb:J}
    \gls{symb:K}
    \gls{symb:L}
    \gls{symb:M}
    \gls{symb:N}
    \gls{symb:O}
    \gls{symb:P}
    \gls{symb:Q}
    \gls{symb:R}
    \gls{symb:S}
    \gls{symb:T}
    \gls{symb:U}
    \gls{symb:V}
    \gls{symb:W}
    \gls{symb:X}
    \gls{symb:Y}
    \gls{symb:Z}
    
    \subsection{Example of Acronmys}
    First use of the Acronym \gls{aA} prints long version with short version in brackets, second and following use will print only the short version: \gls{aA}
    
    \subsection{Special Characters}
    Registered: \TReg\\
    Copyright: \TCop\\
    Trademark: \TTra\\
    
    Großzügig Gréànauôbl Ç.\\
    !"§/()=?öäüß-.,<>\\
    
    \subsection{Example of Lists}
    \begin{itemize}
      \item List entries start with the \verb|\item| command!
      \item Individual entries are indicated with a black dot, a so-called bullet.
      \item The text in the entries may be of any length.
    \end{itemize}
    
    \subsection{Blind Text}
    \Blindtext
    
    \subsection{Example of an Equation}
    \begin{equation}
        \label{eq::Central_Projection}
            \left[\begin{array}{@{}c@{}} {}^c x_{\overline{p}} \\ {}^c y_{\overline{p}} \\ 1 \end{array} \right] =
            \underbrace{\left[\begin{array}{cccc}
                c & 0 & 0 & 0 \\
                0 & c & 0 & 0 \\
                0 & 0 & 1 & 0
            \end{array} \right]}_{\text{projection matrix}}
            \left[\begin{array}{@{}c@{}} {}^k X_p \\ {}^k Y_p \\ {}^k Z_p \\ 1 \end{array} \right] =
            \left[\begin{array}{@{}c@{}} c \cdot {}^k X_p \\ c \cdot {}^k Y_p \\ {}^k Z_p \end{array} \right] =
            \left[\begin{array}{@{}c@{}} c \dfrac{{}^k X_p}{{}^k Z_p} \\ c \dfrac{{}^k Y_p}{{}^k Z_p} \\ 1 \end{array} \right]
    \end{equation}
    
    \subsection{Example of Figure}
    \begin{figure}[ht]
            \centering
            \includegraphics[width=\textwidth]{theory/Pinhole-Camera-Model-Multiple-View-Geometry-in-Computer-Vision.pdf}
            \caption[Mathematical model of a pinhole camera]{Mathematical model of a pinhole camera \cite{Multiple_View_Geometry_in_Computer_Vision}.}
            \label{fig::Pinhole_Camera_Model}
        \end{figure}
    

    \subsection{Example of Reference}
    Reference Figure \ref{fig::Pinhole_Camera_Model} and Equation \ref{eq::Central_Projection}.
    Reference source \cite{3D_introductorytechniques}. Also check out code~\ref{Code:Super Code}.

    \subsection{Example of a Table}
    \begin{table*}[ht]
        \centering
        \caption[Datasets for 3D reconstruction]{Datasets for stereoscopic 3D reconstruction with ground-truth information.}
        \label{Table::Dataset_attributes}
        \begin{tabular}{|c|c|c|c|c|}
            \hline
            ~                        & Dynamic & Truth Type       & Remarks\\
            \hline
            \color{blue} Tsukuba      & --      & Manual           & The first data with GT\\
            \color{orange} Middlebury & --      & Structured Light & Most famous 3D data with GT \\
            \color{red} Hamlyn        & $X$     & partly available & Robotic surgery\\
            \color{cyan} Kitti        & $X$     & Lidar            & Autonomous driving\\
            \color{green} EndoVis     & $X$     & Structured Light & Robotic surgery\\
            \hline
        \end{tabular}
        \begin{tablenotes}
            \small
            \item * This is an example footnote for the table.
        \end{tablenotes}
    \end{table*}

    \subsection{Example of Inserted Code}
    \label{Section::Insert_Code}
    You can use the package listings to import code directly from files. 
    \lstinputlisting[language=C++,
                     caption=Descriptive Caption Text,
                     label=Code:Super Code]
    {theory/cuda_example.cu}

    In addition to this, it is also possible to directly input your code in \LaTeX.
    \begin{lstlisting}[language=C++,
                       caption=Another Descriptive Caption Text,
                       label=Code:Super Code2]
     // Some example
     callFunction();
    \end{lstlisting}

    \subsection{Example of ToDo-Notes}
    \label{Section::ToDo}
    With \verb|\todo{Example ToDo note}| you can define your own ToDo notes.
    \todo{Example ToDo note}

    We prefer the inline ToDo notes, but you can also use floating ones by using the command \verb|\todo[noinline]{Example ToDo note}|.
    % \todo[noinline]{Example ToDo note}

    You can add your ToDos to the Table of Contents with \verb|\todototoc|, or you can print a separate list of ToDo notes with \verb|\listoftodos|.
    If you prefer, you can also just color your text.

    \subsection{Coloring Text}
    \label{Section::Colors}
    You can color text by using predefined commands such as \verb|\red{text}|.
    Colors we provide are as follows: \red{red}, \green{green}, \blue{blue}, and \orange{orange}. Also, feel free to make up your own colors and add them to the \verb|settings+/variables.tex| file. This can be done in a few easy steps:
    \begin{itemize}
        \item Name a color and define it (see \url{https://latexcolor.com/} for ideas)\\
              \verb|\definecolor{dollarbill}{rgb}{0.52, 0.73, 0.4}|
        \item Optional: Create a shorthand command with the new color\\
              \verb|\newcommand\myColor[1]{\textcolor{dollarbill}{\textbf{#1}}}|
        \item Done! Color a text with the new color\\
              \verb|\textcolor{dollarbill}{Text}|\\
              or via shorthand\\
              \verb|\myColor{Text}|
    \end{itemize}

% You can use this to add content for standalone documents if you like
% In this case we would like to show the references.
\ifstandalone
    % Bibliography
    \printbibliography[heading=bibintoc]                         \cleardoublepage

% ----------------------------------------------------------------------------
% Appendix and Glossary
% ----------------------------------------------------------------------------
%     \pagenumbering{Alph} % A, B, C..

% %     % Appendix
%     \input{chapters/appendix}                                          \clearpage

% %     % Symbol list also counts as a glossary object
%     \printglossary[type=main]  % main glossary

% %     % Either print all entries or only used entries for all lists
%     \glsaddallunused
\fi

\end{document}
